\documentclass[conference]{IEEEtran}
\IEEEoverridecommandlockouts
% The preceding line is only needed to identify funding in the first footnote. If that is unneeded, please comment it out.
\usepackage{cite}
\usepackage{amsmath,amssymb,amsfonts}
\usepackage{algorithmic}
\usepackage{graphicx}
\usepackage{textcomp}
\usepackage{xcolor}
\def\BibTeX{{\rm B\kern-.05em{\sc i\kern-.025em b}\kern-.08em
    T\kern-.1667em\lower.7ex\hbox{E}\kern-.125emX}}
\begin{document}

\title{Decentralized Autonomous Rovers\\
{\footnotesize \textsuperscript{*}Freedom Rover Units}
\thanks{}
}

\author{\IEEEauthorblockN{1\textsuperscript{st} Jordy A. Larrea Rodriguez}
\IEEEauthorblockA{\textit{Department of Electrical and Computer Engineering} \\
\textit{University of Utah}\\
Salt Lake City, USA \\
Jordy.larrearodriguez@gmail.com}
\and
\IEEEauthorblockN{2\textsuperscript{nd}  Brittney L. Morales}
\IEEEauthorblockA{\textit{Department of Electrical and Computer Engineering} \\
\textit{University of Utah}\\
Salt Lake City, USA \\
brittneymrls@gmail.com}
\and
\IEEEauthorblockN{3\textsuperscript{rd} Misael Nava}
\IEEEauthorblockA{\textit{Department of Electrical and Computer Engineering} \\
\textit{University of Utah}\\
Salt Lake City, USA \\
misaelnava812@gmail.com}
}

\maketitle

\begin{abstract}
The state of the art in autonomous swarms employs a decentralized model consisting of multi-agent networks. These multi-agent systems hold the potential to adapt to new environments and optimize individual performance to specific tasks without having to deal with global systems prone to single points of failure. Our team's focus lies therein in developing a decentralized multi-agent system. The decentralized multi-agent system will incorporate three to five two-wheel drive rovers interfaced through the Robot Operating System (ROS) by raspberry pi 4 companion computers for edge-computing specifications. A central system is still beneficial for the assignment of global objectives; thus, our proposed decentralized system will employ a central bay station to communicate objectives for the agents to complete (carefully designed demos). LiDAR, simple infrared sensors, and low-resolution cameras will be procured to facilitate real-time (RT) decision-making by the agent(s) in response to the environment. Our development stack will leverage ROS for project management, simulation capabilities, navigation libraries, and native server-client model in robotics applications. The rovers AI will incorporate simultaneous localization and mapping (SLAM) techniques for RT positioning based on a priori grid or map; thus, facilitating navigation through improved state space mapping. Decentralized swarms allow for a set of agents to act as more than distributed actuators: i.e., more like your white blood cells rather than your limb as observed by central systems. Furthermore, drone Swarms hold the potential to gather large quantities of data for monitoring and area mapping that would otherwise prove too costly to collect, and can greatly simplify and reduce manpower in search-and-rescue, disaster recovery, or security scenarios. Swarms essentially hold the capability to replace humans in potentially dangerous and normally costly tasks.
\end{abstract}

\begin{IEEEkeywords}
Decentralized Communication, Swarm Communication, Multi-agent, ROS, Gazebo
\end{IEEEkeywords}

\section{Introduction}
This document is a model and instructions for \LaTeX.
Please observe the conference page limits. 

\section{Background}

\subsection{Maintaining the Integrity of the Specifications}


\section{Methods and Planning}


\subsection{Abbreviations and Acronyms}\label{AA}

\subsection{Units}


\subsection{Equations}


\subsection{\LaTeX-Specific Advice}


\subsection{Some Common Mistakes}\label{SCM}

\subsection{Authors and Affiliations}

\subsection{Identify the Headings}

\subsection{Figures and Tables}

\begin{table}[htbp]
\caption{Table Type Styles}
\begin{center}
\begin{tabular}{|c|c|c|c|}
\hline
\textbf{Table}&\multicolumn{3}{|c|}{\textbf{Table Column Head}} \\
\cline{2-4} 
\textbf{Head} & \textbf{\textit{Table column subhead}}& \textbf{\textit{Subhead}}& \textbf{\textit{Subhead}} \\
\hline
copy& More table copy$^{\mathrm{a}}$& &  \\
\hline
\multicolumn{4}{l}{$^{\mathrm{a}}$Sample of a Table footnote.}
\end{tabular}
\label{tab1}
\end{center}
\end{table}

\begin{figure}[htbp]
\centerline{\includegraphics{fig1.png}}
\caption{Example of a figure caption.}
\label{fig}
\end{figure}

``Temperature/K''.
\section{Discussion}
Add discussion stuff in here.
\section{Demo Description} 
Add demo description stuff in here. 
\section*{Acknowledgment}

\section*{References}

Please number citations consecutively within brackets \cite{b1}. The 
sentence punctuation follows the bracket \cite{b2}. Refer simply to the reference 
number, as in \cite{b3}---do not use ``Ref. \cite{b3}'' or ``reference \cite{b3}'' except at 
the beginning of a sentence: ``Reference \cite{b3} was the first $\ldots$''

Number footnotes separately in superscripts. Place the actual footnote at 
the bottom of the column in which it was cited. Do not put footnotes in the 
abstract or reference list. Use letters for table footnotes.

Unless there are six authors or more give all authors' names; do not use 
``et al.''. Papers that have not been published, even if they have been 
submitted for publication, should be cited as ``unpublished'' \cite{b4}. Papers 
that have been accepted for publication should be cited as ``in press'' \cite{b5}. 
Capitalize only the first word in a paper title, except for proper nouns and 
element symbols.

For papers published in translation journals, please give the English 
citation first, followed by the original foreign-language citation \cite{b6}.

\begin{thebibliography}{00}
\bibitem{b1} G. Eason, B. Noble, and I. N. Sneddon, ``On certain integrals of Lipschitz-Hankel type involving products of Bessel functions,'' Phil. Trans. Roy. Soc. London, vol. A247, pp. 529--551, April 1955.
\bibitem{b2} J. Clerk Maxwell, A Treatise on Electricity and Magnetism, 3rd ed., vol. 2. Oxford: Clarendon, 1892, pp.68--73.
\bibitem{b3} I. S. Jacobs and C. P. Bean, ``Fine particles, thin films and exchange anisotropy,'' in Magnetism, vol. III, G. T. Rado and H. Suhl, Eds. New York: Academic, 1963, pp. 271--350.
\bibitem{b4} K. Elissa, ``Title of paper if known,'' unpublished.
\bibitem{b5} R. Nicole, ``Title of paper with only first word capitalized,'' J. Name Stand. Abbrev., in press.
\bibitem{b6} Y. Yorozu, M. Hirano, K. Oka, and Y. Tagawa, ``Electron spectroscopy studies on magneto-optical media and plastic substrate interface,'' IEEE Transl. J. Magn. Japan, vol. 2, pp. 740--741, August 1987 [Digests 9th Annual Conf. Magnetics Japan, p. 301, 1982].
\bibitem{b7} M. Young, The Technical Writer's Handbook. Mill Valley, CA: University Science, 1989.
\end{thebibliography}
\vspace{12pt}
\color{red}
IEEE conference templates contain guidance text for composing and formatting conference papers. Please ensure that all template text is removed from your conference paper prior to submission to the conference. Failure to remove the template text from your paper may result in your paper not being published.

\end{document}
